\documentclass[12pt]{article}

\usepackage[utf8]{inputenc}
\usepackage[T1]{fontenc}
\usepackage[top=2cm, bottom=2cm, left=2.5cm, right=2.5cm]{geometry}
\usepackage{hyperref}
\usepackage{multirow}
\usepackage{float}
\usepackage{graphicx}
\usepackage{caption}
\usepackage{subcaption}

\sloppy

\begin{document}
%\maketitle

\begin{titlepage}
	\begin{center}
	
	%\begin{figure}[!ht]
	%\centering
	%\includegraphics[width=2cm]{c:/ufba.jpg}
	%\end{figure}

		\large{Universidade Federal de Minas Gerais}\\
		\large{Ensino Básico e Profissional}\\ 
		\large{Colégio Técnico (COLTEC)}\\ 
		\vspace{15pt}
        \vspace{95pt}
        \textbf{\LARGE{Relatório Final de Estágio}}\\
		%\title{{\large{Título}}}
		\vspace{3,5cm}
	\end{center}
	
	\begin{flushleft}
		\begin{tabbing}
			Nome do(a) estagiário(a): Ulisses Drumond Souza Rosa\\
		    Número de matrícula: 2020954529\\
		    Curso e ano de conclusão: Desenvolvimento de Sistemas / 2022\\
		    Endereço e telefone do(a) Estagiário(a): Rua Santo Antônio do Monte 579 / (31) 99300-7834\\
		    Nome da Empresa: Mais 60 Saúde\\ 
		    Setor que desenvolveu o estágio: LifeCode\\
		    Data do Início do Estágio: 11/04/2022\\
		    Total de horas cumpridas no período: 1152\\
	\end{tabbing}
 \end{flushleft}
	\vspace{1cm}
	
	\begin{center}
		\vspace{\fill}
			 Dezembro\\
		 2022
			\end{center}
\end{titlepage}
%%%%%%%%%%%%%%%%%%%%%%%%%%%%%%%%%%%%%%%%%%%%%%%%%%%%%%%%%%%

% % % % % % % % %FOLHA DE ROSTO % % % % % % % % % %

\begin{titlepage}
	\begin{center}
	
	%\begin{figure}[!ht]
	%\centering
	%\includegraphics[width=2cm]{c:/ufba.jpg}
	%\end{figure}

	    \large{Universidade Federal de Minas Gerais}\\
		\large{Ensino Básico e Profissional}\\ 
		\large{Colégio Técnico (COLTEC)}\\ 
\vspace{15pt}
        
        \vspace{85pt}
        
		\textbf{\LARGE{Relatório Final }}
		\title{\large{Título}}
	%	\large{Modelo\\
     %   		Validação do modelo clássico}
			
	\end{center}
\vspace{1,5cm}
	
	\begin{flushright}

   \begin{list}{}{
      \setlength{\leftmargin}{4.5cm}
      \setlength{\rightmargin}{0cm}
      \setlength{\labelwidth}{0pt}
      \setlength{\labelsep}{\leftmargin}}

      \item Relatório de Estágio apresentado à Coordenação do Curso de Desenvolvimento de Sistemas do Colégio Técnico da Universidade Federal de Minas Gerais como requisito parcial para conclusão de curso.

      \begin{list}{}{
      \setlength{\leftmargin}{0cm}
      \setlength{\rightmargin}{0cm}
      \setlength{\labelwidth}{0pt}
      \setlength{\labelsep}{\leftmargin}}
            %ORIENTADOR DA EMPRESA!
			\item Nome do(a) estagiário(a): Ulisses Drumond Souza Rosa\
            \item Professor(a) orientador(a): Lênio Corrêa\

      \end{list}
   \end{list}
\end{flushright}
\vspace{1cm}
\begin{center}
		\vspace{\fill}
		 Dezembro\\
		 2022
			\end{center}
\end{titlepage}
\newpage
% % % % % % % % % % % % % % % % % % % % % % % % % %
\newpage
\tableofcontents
\thispagestyle{empty}

\newpage
\pagenumbering{arabic}
% % % % % % % % % % % % % % % % % % % % % % % % % % %
\section{Apresentação}
Olá, eu me chamo Ulisses, atuo na área de programação, tendo ênfase em desenvolvimento \emph{back-end} e \emph{fullstack}.
\\
Ao longo da minha formação como técnico em Desenvolvimento de Sistemas - no Colégio Técnico da Universidade Federal de Minas Gerais (COLTEC-UFMG) - tive a oportunidade de construir uma sólida lógica de programação. Além disso, aprendi as seguintes linguagens de programação: C, C++, C\#, JavaScript e Java. É importante ressaltar  que eu tenho facilidade para aprender, de maneira autônoma, novas tecnologias e linguagens de programação.
\\
Dentro da área da computação, tenho um grande interesse pelos seguintes tópicos: algoritmos, estruturas de dados e programação concorrente. Vejo esses conceitos como importantes ferramentas, que, em meu dia a dia como programador, me auxiliam na resolução dos mais variados problemas.
\\
Atualmente atuo como estagiário em Desenvolvimento de Software na empresa mais 60 Saúde. A experiência de estagiário tem sido bastante enriquecedora, visto que eu estou tendo a oportunidade de compreender o funcionamento da rotina de um time de desenvolvimento. Ademais, estou colocando em prática vários dos conhecimentos adquiridos em sala de aula como: HTML/CSS/JS, C\#, Banco de Dados e Programação Orientada a Objetos.
\\
Vale salientar que eu sou uma pessoas bastante comunicativa, resiliente e descontraída (nos momentos certos, é claro). Tenho também um enorme interesse pelas diferentes áreas da ciência da computação.

\section{Descrição da empresa}
A mais 60 saúde é uma empresa que tem como principal proposta oferecer antendimento clínico especializado para idosos. Diante disso, através da contratação de geriatras, a empresa busca inovar o mercado de clínica médica.
\\
Atualmente a +60 conta com 3 unidades em Belo Horizonte e 1 em Betim. A mais 60 sáude atende os seguintes planos de saúde: Unimed, Desban, Fundafeemg e Premium Senio. Além disso, a empresa oferece também o +60 Essencial, que é um plano de saúde voltado exclusivamente para pessoas na terceira idade. 
\\
Em 2012, a mais 60 criou uma de suas principais divisões que é o LifeCode. O LifeCode tem como premissa desenvolver um intuitivo sistema de prontuários digitais, para que médicos (e outros funcionários da empresa) consigam monitorar os pacientes da clínica com velocidade e eficiência.

\section{Descrição do setor/área de estágio}

\subsection{LifeCode}
Tendo em vista que meu estágio é em desenvolvimento de software, meu setor de atuação é o LifeCode. Atualmente, eu tenho trabalhado apenas na equipe de desenvolvimento do LifeCode. Até agora, ainda não tive oportunidade de fazer parte do time de análise de dados (Analytcs).
\\
Nesse momento, a equipe do LifeCode conta com 11 programadores. O nosso principal objetivo é dar manutenção e criar novas funcionalidades para o sistema de prontuários digitais da mais 60. É importante mencionar que o time trabalha em sprints (ciclos de 15 dias). Cada tarefa é delegada a um ou mais desenvolvedores e possui um peso (1,2,3,5,8,13) que indica o seu nível de dificuldade.
\\
É importante mencionar que o LifeCode é dividido em 3 sub-sistemas, major, hércules, e panacea. Os dois primeiros formam o back-end e o terceiro o front-end. O servidor é programado em C\# e a interface gráfica, por sua vez, é codificado em JavaScript utilizando o framework React. 

\subsection{A arquitetura da API do LifeCode}
No lifecode existem as entidades que são as classes, cuja função é estruturar os dados do sistema. Os dados das entidades são modificados e acessados por meio de comandos. Ou seja, se nós formos criar uma classe "remédio", teríamos que implementar os comandos: adiciona remédio, remove remédio, altera remédio, e get remédio.
É importante mencionar que cada classe que representa uma entidade tem como campos todas as colunas que essa mesma entidade tem na sua tabela do banco de dados. Por isso que cabe a nós desenvolvedores implementar o mapeamento da informação que está armazenada no banco de dados para as classes que estão implementadas na linguagem C\#.


\section{Objetivo do Estágio}
O principal objetivo do estágio foi desenvolver e aplicar os conhecimentos obtidos no curso de Desenvolvimento de Sistemas do Colégio Técnico da Universidade Federal de Minas Gerais. Ademais, o estágio também tem a finalidade de me introduzir no mercado de trabalho de programação e tecnologia.

\section{Descrição das atividades desenvolvidas}
\subsection{Treinamento}
Antes de receber tarefas, eu fui submetido a um processo de treinamento. Durante aproximadamente 1 mês, eu fiquei estudando cursos on-line sobre os mais variados assuntos (Fundamentos de C\#, LINQ, Programação assíncrona, Rest-API, Design Patterns, React JS, CSS, Redux). O período foi de grande aprendizado, tendo em vista que eu aprofundei	 os meus conhecimentos no 
desenvolvimento de aplicações para o ambiente web.
\\
Além disso, nesse curto período os desenvolvedores Thiago e Raquel mostraram para mim todo o funcionamento do LifeCode, e as ideias por trás desse complexo sistema.


\subsection{Minha Primeira Tarefa}
A minha primeira tarefa no LifeCode foi alterar o formulário de encaminhamento. Era requisitado que eu criasse um novo tipo de encaminhamento (Urgência e Emergência) e realizasse as devidas alterações no sistema, para que fosse possível aplicar as outras funcionalidades do LifeCode aos encaminhamentos de Urgência e Emergência. Vale a pena destacar que a minha tarefa foi \emph{fullstack}, ou seja, eu alterei tanto o \emph{back-end} quanto o \emph{front-end}.


\subsection{Encaminhamento para Profissional Externo}
A minha segunda tarefa foi ainda no formulário de encaminhamento. Foi requisitado que eu adicionasse um novo tipo de encaminhamento ao formulário em análise, que nesse caso é o de profissional externo (profissionais que são parceiros da mais 60). Essa não foi uma tarefa muito difícil na parte da API, visto que era necessário apenas adicionar um novo item no enum de tipos de encaminhamentos. A parte da UI também foi tranquila, eu apenas precisei de colocar um novo item no select do formulário de encaminhamento.


\subsection{Profissional Externo Ativo/Inativo}
Adicionei o campo "ativo" a entidade profissional externo, que, por sua vez, indica se um profissional externo está em atividade. Vale mencionar que não será possível encaminhar pacientes para profissionais que estão inativos. A minha principal dificuldade nessa tarefa foi implementar a função que realiza o filtro de profissionais que são ativos e mostrar esses profissionais filtrados na UI da maneira certa.

\subsection{Identificador Externo}
Essa foi a tarefa mais simples que eu fiz até agora. Nela eu precisei apenas de criar um novo campo (identificador externo) na entidade paciente. Esse novo campo salva o identificador (do sistema da Unimed) de um paciente da mais 60 que também possui convênio na Unimed;

\subsection{Lista de espera}
A tarefa que eu estou trabalhando atualmente é a mais difícil que eu desde que o estágio começou, já que eu preciso de implementar uma nova entidade para o sistema. Nesse caso, eu estou implementando uma lista de espera, para que seja possível criar uma fila de pessoas que querem ser atendidas por um médico que já está com a agenda cheia. Até agora, foi implementada apenas a parte da API, ou seja os comandos para inserção, busca e remoção de um item da fila de espera.

\subsection{Modificação no encaminhamento}
A minha última tarefa no lifecode foi realizar uma mudança na entidade de encaminhamento na API, agora ela segue um novo padrão que foi decidido pelos principais médicos da Mais 60. Essa foi uma grande mudança no sistem, uma vez que grande parte do código do encaminhamento passou a ficar depreciado, ou seja, eu precisei de refatorar um grande volume de código e alterar inúmeras tabelas do banco de dados. 



\section{Conceitos abordados durante o estágio}
Durante o estágio inúmeros conceitos foram trabalhados, dentre eles os mais importantes foram git, metodologias ágeis, desenvolvimento back-end e desenvolvimento front-end.
\\
A ferramenta de versionamento de código utilizada é o git, graças a esse software a equipe do lifecode conseguia trabalhar em conjunto sem ocorrer severos conflitos de versão. No que se refere às tecnologias utilizadas para o desenvolvimento web, a equipe utilizava o .Net framework para o desenvolvimento da API e o framework React JS para o desenvolvimento da UI.

\section{Descrição da rotina de trabalho e área de atuação}
A equipe lifecode, como mencionado anteriormente, trabalha com base em metodologias ágeis. Nesse sentido, realizamos reuniões diárias e quinzenalmente realizamos reuniões de sprint. Dessa maneira, a minha jornada de trabalho começava com a reunião diária, e na sequência continuava com o prosseguimento das tarefas que eram atribuídas.
\\
Agora falando sobre a área de atuação do lifecode, a equipe de desenvolvedores da Mais 60 atua no desenvolvimento e na manutenção de um sistema de prontuários digitais que será utilizado pelos médicos e outros funcionários das clínicas da rede Mais 60.

\section{Discriminação dos equipamentos utilizados no estágio}
O equipamento utilizado no estágio foi o notebook Dell Inspiron I15 (processador: I7 11º geração; 16 GB memória RAM; placa de vídeo: NVIDIA® GeForce® MX450, 2GB GDDR5; 512 GB SSD).
\\
O notebook foi essencial para a realização das atividades do estágio, visto que a execução de todas as partes do lifecode exige um hardware consideralvelmente forte.
\\
Além disso, para o desenvolvimento, eu utilizei as IDE's da JetBrains Rider e WebStorm. 

\section{Aprimoramento da vida profissional}
O estágio na Mais 60 foi uma experiência bastante enriquecedora, uma vez que eu evolui bastante como desenvolvedor. Durante esse período como estagiário, eu aprendi como que uma empresa funciona e principalmente qual é a maneira correta de desenvolver uma aplicação em grupo.
\\
Ademais, eu assimilei inúmeras boas práticas de programação e entendi de fato o funcionamento de uma aplicação web. Agora eu estou em plenas condições de implementar um sistema web que conta com uma interface grafica que se comunica com uma API. 

\section{Condições de trabalho}
As condições de trabalho do meu estágio foram muito boas. Toda a equipe do lifecode sempre foi muito disposta a ajudar os estagiários sempre quando eles deparavam com problemas e dúvidas. Outrossim, não houve nenhum tipo de pressão no que se refere à entrega de tarefas no prazo estipulado.
\\ 
Toda a equipe do lifecode sempre foi muito compreensiva, o que foi trivial para a criação de um ambiente de trabalho bastante tranquilo e flexível.

\section{Análise comparativa do curso e do estágio}
O Curso de Técnico em Desenvolvimento de Sistemas forneceu importantes fundamentos na área de programação e computação. No COLTEC, eu tive contato com assuntos importantes assuntos, que, por sua vez, são amplamente requisitados no mercado de trabalho, como, algoritmos, estruturas de dados e código limpo. Com isso, eu montei um sólido currículo de programador.
\\
Entretanto, apesar do COLTEC fornecer um ensino de excelência, o curso de Desenvolvimento de Sistemas deixa de lado importantes conceitos de programação para web, a título de exemplo, frameworks de Java Script e REST API's. Dessa maneira, em um primeiro momento eu tive bastante dificuldade para adaptar às tecnologias utilizadas no meu trabalho.
\\
É importante ressaltar que a escolha da linguagem C como primeira linguagem de programação a ser ensinada no COLTEC, uma vez que ela contribui para a construção de uma forte lógica computacional.

\section{Conclusão}
Por fim, concluo que o estágio foi uma experiência de grande aprendizado, pois eu finalmente entendi como que uma empresa de software funciona. Além disso, eu aprendi a utilizar importantes teconlogias como React JS, C\# e MySQL server, o que, por sua vez, favorece um futuro ingresso no mercado de trabalho. 

\end{document}
