\documentclass[11pt]{article}
\usepackage[utf8]{inputenc}
\usepackage[brazilian]{babel}
\usepackage{hyperref}
\usepackage{lipsum}
\usepackage{geometry}
\usepackage[skip=5pt plus1pt, indent=20pt]{parskip}

\usepackage{amsthm,amssymb,amsmath}

\usepackage{listings}
\usepackage{xcolor}

\definecolor{codegreen}{rgb}{0,0.6,0}
\definecolor{codegray}{rgb}{0.5,0.5,0.5}
\definecolor{codepurple}{rgb}{0.58,0,0.82}
\definecolor{backcolour}{rgb}{0.95,0.95,0.92}

\lstdefinestyle{mystyle}{
    backgroundcolor=\color{backcolour},   
    commentstyle=\color{codegreen},
    keywordstyle=\color{magenta},
    numberstyle=\tiny\color{codegray},
    stringstyle=\color{codepurple},
    basicstyle=\ttfamily\footnotesize,
    breakatwhitespace=false,         
    breaklines=true,                 
    captionpos=b,                    
    keepspaces=true,                 
    numbers=left,                    
    numbersep=5pt,                  
    showspaces=false,                
    showstringspaces=false,
    showtabs=false,                  
    tabsize=2
}

\lstset{style=mystyle}

\newtheorem*{theorem}{Theorem}

\newcommand{\NN}{\mathbb{N}}
\newcommand{\ZZ}{\mathbb{Z}}
\newcommand{\RR}{\mathbb{R}}
\newcommand{\QQ}{\mathbb{Q}}
\newcommand{\CC}{\mathbb{C}}

\geometry{margin=2.5cm}

\title{\textbf{Trabalho Prático De Programação e Desenvolvimento de Software I}}
\author{\textbf{Ulisses Drumond Souza rosa}}
\date{\parbox{\linewidth}{\centering%
    Universidade Federal de Minas Gerais (UFMG)\endgraf
    Belo Horizonte - MG - Brasil\endgraf\bigskip
    \href{mailto:email@ufmg.br}{ulissao@proton.me}}}

\begin{document}

\maketitle

%%%%%%%%%%%%%%%%%%%%%%%%%%%%%%%%%%%%%%%%%%%%%%%%%%%%%%%%%%%%%%%%%%%%%%%%%%%%%%

\section{Introdução}

    \par Essa é a documentação do \href{https://github.com/ullas-college-projects/Pds1-pratical-work}{trabalho prático} da disciplina PDS I. 
    \par A tarefa desse trabalho consistia em ler um input de texto que descrevia as informações de um determinado número de sistemas solares, armazena-los em um array, ordenar o array e, por fim, imprimir o nome dos sistemas solares. Para realizar tal tarefa, nós utilizamos o "bromerosort" (A.K.A quicksort).

%%%%%%%%%%%%%%%%%%%%%%%%%%%%%%%%%%%%%%%%%%%%%%%%%%%%%%%%%%%%%%%%%%%%%%%%%%%%%%

\section{Implementação}

    \par Para solucionar o problema descrito no trabalho prático eu criei 5 arquivos: main.c, input\_reader.c, input\_reader.h, solar\_system.c e solar\_system.h. É importante lembrar que esse trabalho prático foi implementado na linguagem de programação C. 

    \subsection{Input Reader}
        \par O módulo "input\_reader" tem como função ler os inputs que o usuário digitar e construir os objetos na memória a partir dos dados inseridos. O módulo em questão possui as seguintes funções.

        \par
        \lstinputlisting[language=C]{input_reader.h}

        \par Essas funções leem os inputs e salvam os valores providos pelo usuário na memória do programa.

    \subsection{Solar System}
        \par Nesse módulo, eu fiz a definição dos tipos: moon\_t, planet\_t e solar\_system\_t, que são os registros que serão armazenados e processados pelo programa. É importante ressaltar que nesse projeto não há uso de alocação estática, toda alocação de memória que eu fiz no trabalho é dinâmica. Além disso, nesse módulo há também a implementação da função 'greater\_than` que verifica se o primeiro sistema solar é mais "valioso" que o segundo. Temos também as funções quicksort e partition, que basicamente implementam o bromero sort.

        

%%%%%%%%%%%%%%%%%%%%%%%%%%%%%%%%%%%%%%%%%%%%%%%%%%%%%%%%%%%%%%%%%%%%%%%%%%%%%%

\section{Instruções de Compilação e execução}

    \par É importante ressaltar que essa documentação é baseada nesse \href{https://github.com/ullas-college-projects/Pds1-pratical-work}{projeto}. Como só é permitido enviar apenas um arquivo no moodle, eu juntei o conteúdo de todos os arquivos no arquivo "main2.c". Mas para ficar mais fácil de explicar eu documentei o projeto como se ele fosse dividido em vários módulos.

    \lstinputlisting[language=bash]{bash.bash}

%%%%%%%%%%%%%%%%%%%%%%%%%%%%%%%%%%%%%%%%%%%%%%%%%%%%%%%%%%%%%%%%%%%%%%%%%%%%%%%

\section{Conclusão}

    \par Esse trabalho prático foi bastante interessante pois implementamos um dos algoritmos de ordenação mais famosos, o quicksort. Além disso deu pra colocar em prática todos os conhecimentos adiquiridos na disciplina PDS I. 

    \par Gostaria de deixar claro que eu não coloquei tantos comentários no código pois eu acredito que a forma com que o meu código está bem autoexplicativo (as variáveis e funções estão com nomes adequados) e modularizado (cada função faz o que está descrito no seu nome).

%%%%%%%%%%%%%%%%%%%%%%%%%%%%%%%%%%%%%%%%%%%%%%%%%%%%%%%%%%%%%%%%%%%%%%%%%%%%%%%



\end{document}

