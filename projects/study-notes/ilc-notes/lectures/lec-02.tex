
\lesson{Oct 19 2022 Wen (12:28:10)}{Predicate Logic}

\subsection{Introduction to predicate logic}
\label{sub_sec:sub_section_2}

\par Unfortunately, the propositional logic can not represent every mathematical affirmation. However, with predicate logic we can construct "mathematical phrases" such as "Every computer science student is nerd". \\

\begin{note}
    \par A predicate is a propositional functional
    \begin{example}
        Let $P(x):$ "x is a real number"
        \begin{itemize}
            \item $P(0) = true$ 
            \item $P(\sqrt{-1}) = false$ 
        \end{itemize}
    \end{example}
\end{note}

\vspace{0.5cm}

\par $ \forall x: P(x) \Rightarrow$ Universal quantifier. That is, for every value in the domain the predicate P(x) is true.

\par $ \exists x: P(x) \Rightarrow$ Existencial quantifier. That is, for at least one value of the domain the predicate P(x) is true. 

\begin{example}
    \par Quantifiers: \\
    \begin{itemize}
        \item $ \forall x \in \mathbb{Z}: x^2 > 0$
        \item b 
    \end{itemize}
\end{example}





\begin{theorem}
This is a theorem.
\end{theorem}
\begin{proof}
This is a proof.
\end{proof}
\begin{example}
This is an example.
\end{example}
\begin{explanation}
This is an explanation.
\end{explanation}
\begin{claim}
This is a claim.
\end{claim}
\begin{corollary}
This is a corollary.
\end{corollary}
\begin{prop}
This is a proposition.
\end{prop}
\begin{lemma}
This is a lemma.
\end{lemma}
\begin{question}
This is a question.
\end{question}
\begin{solution}
This is a solution.
\end{solution}
\begin{exercise}
This is an exercise.
\end{exercise}
\begin{definition}[Definition]
This is a definition.
\end{definition}
\begin{note}
This is a note.
\end{note}

% subsection sub_section_2 (end)

\newpage