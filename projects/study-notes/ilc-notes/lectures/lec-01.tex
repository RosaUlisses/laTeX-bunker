\lesson{Oct 17 2022 Mon (12:28:10)}{Propositional Logic}

\par The logic is the path of the philosophy that deals with valid inferences and the preservation of the truth. With logic we can reach truthy conclusions.
The propositional logic is based in the principle of the third excluded. That is, the value of a proposition can be only be true or false.

\begin{note}
    \begin{center}
       \par $
            p \rightarrow q
        $ (p is suficient and q is necessary)
        \hspace{3.5em}
        \begin{tabular} {| m{1cm} | m{1cm} | m{1cm} |}
        \hline $p$ & $q$ & $p \rightarrow q$ \\
        \hline $T$ & $T$ & $T$ \\ 
        \hline $T$ & $F$ & $F$ \\ 
        \hline $F$ & $T$ & $T$ \\ 
        \hline $F$ & $F$ & $T$ \\
        \hline 
        \end{tabular}
    \end{center}

    \begin{itemize}
        \item opposite: $q \rightarrow p$
        \item inverse: $\lnot p \rightarrow \lnot q$
        \item contra-positive: $\lnot q \rightarrow \lnot p$
    \end{itemize}  
\end{note}

\par A consistent system is a system that has a true value on its truth table.

\begin{itemize}
    \item tautology: A expression that is always true
    \item contradiction: A expression that is always false
    \item contingency: A expression that is not a tautoloy and it is not a contradiction
\end{itemize}

\begin{theorem}
    De Morgan's Law
    \begin{itemize}
        \item $\lnot (p \lor q) \equiv \lnot p \land \lnot q$        
        \item $\lnot (p \land q) \equiv \lnot p \lor \lnot q$
    \end{itemize}   
\end{theorem}

\par A expression is satisfable only if has at least one true on its truth table. It is important to say that every computational problem that can solved using a algorithm, can be reduced to the satisfation problem.

\newpage