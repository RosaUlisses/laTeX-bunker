
\lesson{Oct 19 2022 Wen (12:28:10)}{Analytic Geometry}

\subsection{Planes}
\label{sub_sec:sub_section_2}

\begin{definition}
  In the space, the equation of a plane is determined by a normal vector and a point.
\end{definition}

\begin{theorem}
  A equation of plane $\pi$ that has the point $p_{\scriptstyle{0}} = (x_{\scriptstyle{0}}, y_{\scriptstyle{0}}, z_{\scriptstyle{0}})$
  and has a orthogonal vector $N = (a, b, c)$ is defined by \\

  $\pi : ax + by + cz + d = 0$ \\

  $d = -ax_{\scriptstyle{0}} -by_{\scriptstyle{0}} -cz_{\scriptstyle{0}}$
\end{theorem}

\begin{note}
  \begin{itemize}
    \item $p \in \pi \land q \in \pi \rightarrow \vec{pq} \parallel \pi$
    \item $p \in \pi \land q \in \pi \rightarrow (p \times q) \perp \pi$
  \end{itemize}
\end{note}

\begin{definition}
  Let $p_{\scriptstyle{0}} = (x_{\scriptstyle{0}}, y_{\scriptstyle{0}}, z_{\scriptstyle{0}})$ in the plane $\pi$, and two not colinear vectors
  $v = (v_{\scriptstyle{1}}, v_{\scriptstyle{2}}, v_{\scriptstyle{3}})$ and $w = (w_{\scriptstyle{1}}, w_{\scriptstyle{2}}, w_{\scriptstyle{3}})$ so: \\

  $p = (x, y, z) \in \pi \leftrightarrow \vec{p_{\scriptstyle{0}}p} = \alpha v + \beta w \;$ so:  \\
  $\vec{p_{\scriptstyle{0}}p} = (x - x_{\scriptstyle{0}}, y - y_{\scriptstyle{0}}, z - z_{\scriptstyle{0}}) = \alpha \cdot (v_{\scriptstyle{1}}, v_{\scriptstyle{2}}, v_{\scriptstyle{3}}) + \beta \cdot (w_{\scriptstyle{1}}, w_{\scriptstyle{2}}, w_{\scriptstyle{3}})$ \\

  $\begin{bmatrix}
    x \\
    y \\
    z \\
  \end{bmatrix} = 
  \begin{bmatrix}
    x_{\scriptstyle{0}} \\
    y_{\scriptstyle{0}} \\
    z_{\scriptstyle{0}} \\
  \end{bmatrix}
  +
  \begin{bmatrix}
    v_{\scriptstyle{1}} \\
    v_{\scriptstyle{2}} \\
    v_{\scriptstyle{3}} \\
  \end{bmatrix} \cdot \alpha
  + 
  \begin{bmatrix}
    w_{\scriptstyle{1}} \\
    w_{\scriptstyle{2}} \\
    w_{\scriptstyle{3}} \\
  \end{bmatrix}  \cdot \beta $
\end{definition}

\begin{theorem}
  If $\pi_{\scriptstyle{1}}$ and $\pi_{\scriptstyle{2}}$ are two planes with normal vectors $N_{\scriptstyle{1}}, N_{\scriptstyle{2}}$,
  so the angle between $\pi_{\scriptstyle{1}}$ and $\pi_{\scriptstyle{2}}$ is defined by: \\

  $cos \theta = \frac{N_{\scriptstyle{1}} \cdot N_{\scriptstyle{2}}}{\|N_{\scriptstyle{1}}\| \cdot \|N_{\scriptstyle{2}}\|}$
\end{theorem}

\begin{theorem}
This is a theorem.
\end{theorem}
\begin{proof}
This is a proof.
\end{proof}
\begin{example}
This is an example.
\end{example}
\begin{explanation}
This is an explanation.
\end{explanation}
\begin{claim}
This is a claim.
\end{claim}
\begin{corollary}
This is a corollary.
\end{corollary}
\begin{prop}
This is a proposition.
\end{prop}
\begin{lemma}
This is a lemma.
\end{lemma}
\begin{question}
This is a question.
\end{question}
\begin{solution}
This is a solution.
\end{solution}
\begin{exercise}
This is an exercise.
\end{exercise}
\begin{definition}[Definition]
This is a definition.
\end{definition}
\begin{note}
This is a note.
\end{note}

% subsection sub_section_2 (end)

\newpage
