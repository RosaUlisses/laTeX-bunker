\lesson{Oct 17 2022 Mon (12:28:10)}{Analytic Geometry}

\subsection{Vectors}
\label{sub_sec:sub_section_1}

\par A vector is a mathematical object that has lenght and direction.

\begin{theorem}
  \hspace{10em}
  \begin{enumerate}
    \item $ v + w = w + v$
    \item $v + (w + u) = (v + w) + u$
    \item $v+ (-v) = 0$
    \item $v+ \vec{0} = 0$
    \item $\alpha (w + u) = \alpha w + \alpha u$
    \item $(\alpha + \beta) \cdot u = \alpha u + \beta u$
    \item $(\alpha \cdot \beta) \cdot u = \alpha (u \cdot \beta u)$
    \item $1 \cdot u = u$
  \end{enumerate}
\end{theorem}

\vspace{3em}
\par The lenght of a vector is called norm. Let $V = (v_{\scriptstyle{1}}, v_{\scriptstyle{2}})$. So $\|V\| = \sqrt{v_{\scriptstyle{1}} ^ 2 + v_{\scriptstyle{2}}^2}.$

\begin{note}
  \begin{itemize}
    \item $distance(p, q) = \| \vec{pq} \|$
    \item $\|\alpha v\| = | \alpha | \cdot \| v \|$
  \end{itemize}
\end{note}

\begin{definition}
  $\| v \| = 1 \rightarrow v$ is a unitary vector \\
  \begin{flalign*}
    u = \frac{1}{\| v \|} \cdot v \\ 
    \| u \| = \| \frac{1}{ \| v \|} \cdot v \| \\
    \| u \| = | \frac{1}{\| v \|} | \cdot \| v \| = 1
  \end{flalign*}
\end{definition}

\begin{definition}
  The dot product of two vectors v and w is defined by: \\ \\
  $v \cdot w = v_{\scriptstyle{1}} \cdot w_{\scriptstyle{1}} + ... + v_{\scriptstyle{n}} \cdot w_{\scriptstyle{n}}$
\end{definition}

\begin{theorem}
  \hspace{10em}
  \begin{enumerate}
    \item $u \cdot v = v \cdot u$
    \item $u \cdot (v + w) = u \cdot v + u \cdot w$
    \item $\alpha (u \cdot v) = (\alpha \cdot u) \cdot u = u \cdot (\alpha \cdot v)$
    \item $v \cdot v = \|v\| ^ 2$
    \item $\forall w: w \cdot w = 0 \leftrightarrow w = \vec{0}$
  \end{enumerate}
\end{theorem}

\begin{theorem}
  $\cos \theta = \frac{v \cdot w}{\|v\| \cdot \|w\|}$ 
  \begin{enumerate}
    \item $0^\circ \leq \theta < 90^\circ =  v \cdot w > 0$ 
    \item $\theta = 90^\circ \leftrightarrow v \cdot w = 0$ ($v$ and $w$ are orthogonal) 
    \item $0^\circ < \theta \leq 180^\circ \leftrightarrow v \cdot w < 0$
  \end{enumerate}
\end{theorem}

\begin{definition}
  Given two vectors $v$ and $w$, the orthogonal projection, $proj_{\scriptstyle{w}}v$, of $v$ over $w$ is the vector paralel to w that $v - proj_{\scriptstyle{w}}v$ orthogonal to w. 
  \begin{theorem}
    $proj_{\scriptstyle{w}}v = (\frac{v \cdot w}{\| w \|^2}) \cdot w$
  \end{theorem}
\end{definition}

\vspace{10em}

\subsection{Vectors in the space}

\begin{definition}
  Let $v = (v_{\scriptstyle{1}}, v_{\scriptstyle{2}}, v_{\scriptstyle{3}})$ and $w = (w_{\scriptstyle{1}}, w_{\scriptstyle{2}}, w_{\scriptstyle{3}})$ two vectors in the space. 
  So the cross product between $v$ and $w$ is:

  $v \times w = \left( 
    \begin{vmatrix} v_{\scriptstyle{2}} & v_{\scriptstyle{3}} \\ w_{\scriptstyle{2}} & w_{\scriptstyle{3}} \end{vmatrix},
   -\begin{vmatrix} v_{\scriptstyle{1}} & v_{\scriptstyle{3}} \\ w_{\scriptstyle{1}} & w_{\scriptstyle{3}} \end{vmatrix},
    \begin{vmatrix} v_{\scriptstyle{1}} & v_{\scriptstyle{2}} \\ w_{\scriptstyle{1}} & w_{\scriptstyle{2}} \end{vmatrix}
   \right) $ 
\end{definition}

\begin{theorem}
  \hspace{10em}
  \begin{enumerate}
    \item $v \times w = -(w \times v)$
    \item $v \times w = \vec{0} \leftrightarrow v \parallel w$
    \item $(v \times w) \perp v \land (v \times w) \perp w$
    \item $v \cdot v = \|v\| ^ 2$
    \item $v \times (w + u) = (v \times w) + (v \times u)$
  \end{enumerate}
\end{theorem}

\pagebreak
\begin{note}
  The canonical vectors $\vec{l} = (1,0,0)$, $\vec{j} = (0,1,0)$ and $\vec{k} = (0,0,1)$ are unit vectors paralel to the coordinate axes.
  They are very important because every vector in the space is a linear cobination of the unit vectors \\

  $\forall v \in \mathbb{R}_{\scriptstyle{3}}: v = \alpha \vec{l} + \beta \vec{j} + \lambda \vec{k}; \; \alpha, \beta, \lambda \in \mathbb{R}_{\scriptstyle{3}}$

\end{note}

\begin{theorem}
  \hspace{10em}
  \begin{enumerate}
    \item $\|v \times w\| = \| v \| \cdot \| w \| \cdot \sin \theta$
    \item The direction is of $v \times w$ is orthogonal to $v$ and $w$
  \end{enumerate}

  \par The area of a paralelogram determined by two vectors is $\|v \times w \|$

  \begin{definition}
    $(v \times w) \cdot u = 
    \begin{vmatrix}
      v_{\scriptstyle{1}} & v_{\scriptstyle{2}} & v_{\scriptstyle{3}} \\ 
      w_{\scriptstyle{1}} & w_{\scriptstyle{2}} & w_{\scriptstyle{3}} \\ 
      u_{\scriptstyle{1}} & u_{\scriptstyle{2}} & u_{\scriptstyle{3}} \\ 
    \end{vmatrix}$

    \begin{note}
      The volume of paralelephiped determined by the vectors ${v, w, u}$ in the space is equal to $|(v \times w) \cdot u|$
    \end{note}

    \begin{corollary}
      The vectors ${v, w, u}$ are in the same plane if and only if $(v \times w) \cdot u = 0$
    \end {corollary}
  \end{definition}
\end{theorem}

\newpage
